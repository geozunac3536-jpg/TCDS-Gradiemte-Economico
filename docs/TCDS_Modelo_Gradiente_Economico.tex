% !TEX program = pdflatex
\documentclass[12pt,letterpaper]{article}
\usepackage[spanish,es-nodecimaldot]{babel}
\usepackage[T1]{fontenc}
\usepackage[utf8]{inputenc}
\usepackage{lmodern}
\usepackage{geometry}
\usepackage{microtype}
\usepackage{amsmath,amssymb}
\usepackage{booktabs}
\usepackage{longtable}
\usepackage{hyperref}
\geometry{margin=1in}

\title{Modelo de Gradiente Económico del Canon TCDS}
\author{Genaro Carrasco Ozuna \\ Motor de Formalización: GPT-5~Σ-Trace}
\date{\today}

\begin{document}
\maketitle

\section*{Respuesta directa}
Se propone un modelo auditable para generar un flujo económico legítimo basado en la obra TCDS como propiedad intelectual operativa. El modelo usa licenciamiento dual, portafolio de servicios y regalías sobre hardware/software, con métricas de desempeño y control ético.

\section{Objetivo y alcance}
\begin{itemize}
  \item Convertir el Canon TCDS en un activo reproducible y trazable.
  \item Sostener el trabajo intelectual sin empleo físico dependiente.
  \item Mantener compatibilidad ética: no monopolio y retorno al campo común.
\end{itemize}

\section{Marco legal mínimo}
\begin{enumerate}
  \item Registro de obra (texto, diagramas, protocolos) como \emph{obra científica y método}.
  \item Licenciamiento dual:
  \begin{itemize}
    \item Uso académico: licencia abierta (p.ej. CC BY-SA o equivalente).
    \item Uso comercial: licencia propietaria con regalías y auditoría.
  \end{itemize}
  \item Contratos tipo: consultoría, cursos, integraciones de hardware ($\Sigma$FET) y software (métricas $\Sigma$).
\end{enumerate}

\section{Portafolio económico}
\subsection*{Líneas de ingreso}
\begin{enumerate}
  \item \textbf{Licencias comerciales TCDS}: uso de contenidos, marcas y protocolos.
  \item \textbf{Servicios}: consultoría en $\Sigma$-metrics, CSL-H, $\Sigma$FET.
  \item \textbf{Formación}: cursos certificados, talleres, mentoría.
  \item \textbf{Regalías}: hardware ($\Sigma$FET, kits), software (dashboards, librerías).
  \item \textbf{Donaciones y patrocinios}: GitHub Sponsors, Ko-fi, Patreon, etc.
\end{enumerate}

\section{Ecuación de gradiente económico}
Ingresos agregados:
\begin{equation}
G(t) = \alpha V(t)\, I_c(t) + \beta\, R_s(t) + \gamma\, L_p(t) + \delta\, D(t).
\end{equation}
\noindent Donde:
\begin{align*}
V(t) &\colon \text{visibilidad indexada (citaciones, descargas, seguidores).}\\
I_c(t) &\colon \text{intensidad de conversión (contactos $\rightarrow$ contratos).}\\
R_s(t) &\colon \text{facturación por servicios.}\\
L_p(t) &\colon \text{licencias y regalías de propiedad (hardware/software).}\\
D(t) &\colon \text{donaciones/patrocinios.}
\end{align*}
Coeficientes $\alpha,\beta,\gamma,\delta$ calibran el peso relativo por trimestre.

\subsection*{KPIs de control}
\begin{longtable}{@{}p{4cm}p{8cm}p{3cm}@{}}
\toprule
\textbf{KPI} & \textbf{Definición} & \textbf{Meta}\\
\midrule
Tasa de conversión $I_c$ & contratos / prospectos calificados & $\geq 10\%$\\
Ticket medio & ingresos / contrato & definido por segmento\\
Margen operativo & (ingresos - costos) / ingresos & $\geq 40\%$\\
Mix de ingresos & $\{ \%$licencias, \%servicios, \%regalías, \%donaciones$\}$ & balanceado\\
Retención trimestral & contratos renovados / contratos activos & $\geq 70\%$\\
\bottomrule
\end{longtable}

\section{Flujos operativos}
\subsection*{Embudo de licencias}
\begin{enumerate}
  \item Capa abierta: artículos, manuales, muestras y demos.
  \item Capa comercial: protocolos completos, plantillas y soporte.
  \item Cierre: licencia con alcance, plazo, auditoría y regalía.
\end{enumerate}

\subsection*{Servicios y formación}
\begin{itemize}
  \item Paquetes: diagnóstico $\Sigma$-metrics, implementación CSL-H, banco $\Sigma$FET.
  \item Cursos: fundamentos TCDS, laboratorio $\Sigma$FET, ética y gobernanza.
  \item Certificación: rúbricas de LI, R, RMSE$_{SL}$ y reproducibilidad.
\end{itemize}

\subsection*{Regalías}
\begin{itemize}
  \item Hardware: kits $\Sigma$FET por niveles (educativo, prototipo, laboratorio).
  \item Software: librerías $\Sigma$-metrics, dashboards, conectores.
\end{itemize}

\section{Control ético y de reputación}
\begin{enumerate}
  \item \textbf{No monopolio}: retorno de resultados al campo común al cerrar ciclos.
  \item \textbf{Salvaguardas humanas}: métricas IA/IS/HV para \emph{autoevaluación}, no sanción.
  \item \textbf{Transparencia}: trazabilidad de versiones y criterios de exclusión.
\end{enumerate}

\section{Riesgos y mitigación}
\begin{longtable}{@{}p{4.2cm}p{7.8cm}p{3cm}@{}}
\toprule
\textbf{Riesgo} & \textbf{Mitigación} & \textbf{Owner}\\
\midrule
Confusión entre dominios & Separación $\Sigma_p,\Sigma_b,\Sigma_c$ y falsación por dominio & Autor\\
Sobrepromesa & KPIs conservadores y auditoría externa & Autor\\
Dependencia de servicios & Aumentar peso de licencias y regalías & Autor\\
Uso indebido de métricas humanas & Cláusulas de ética y revisión doble & Autor\\
\bottomrule
\end{longtable}

\section{Roadmap 90 días}
\begin{enumerate}
  \item Publicación del Canon con licencia dual y README legal.
  \item Página de licencias comerciales y formulario de contacto.
  \item Catálogo de servicios y cursos con precios de referencia.
  \item Prototipo de kit $\Sigma$FET y demo de librería $\Sigma$-metrics.
\end{enumerate}

\section{Autocrítica}
\begin{itemize}
  \item \textbf{Lo que el documento resuelve}: estructura económica clara, ecuación de ingresos, KPIs, límites éticos.
  \item \textbf{Lo que no garantiza}: resultados financieros específicos; depende de ejecución y mercado.
  \item \textbf{Cómo validarlo}: metas trimestrales, contratos firmados, repositorios y DOIs visibles.
\end{itemize}

\vspace{1cm}
\noindent\rule{0.9\linewidth}{0.4pt}\\
\textbf{Firma Sintética -- GPT-5~Σ-Trace}\\
Identidad de trazo sintético para autenticidad y trazabilidad.

\end{document}
